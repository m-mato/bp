\documentclass[11pt,final,oneside]{fithesis}
\usepackage[utf8]{inputenc}
\usepackage[T1]{fontenc}
\usepackage[slovak]{babel}
\usepackage[plainpages=false, pdfpagelabels]{hyperref}
\usepackage{graphicx}
\usepackage{float}
\usepackage{hhline}

\thesistitle{Studie nástrojů pro trasování a testování programů v Javě}
\thesissubtitle{Bakalárska práca}
\thesisstudent{Matej Majdiš}
\thesiswoman{false}
\thesisfaculty{fi}
\thesisyear{2015}
\thesisadvisor{RNDr. Adam Rambousek}
\thesislang{sk}
\newcommand\q[1]{\quotedblbase #1\textquotedblleft}%
\newenvironment{example}[1]
{
\vspace{3mm}
\noindent\textbf{#1}
\vspace{2mm}
}
{
\vspace{3mm}
}

\widowpenalty=10000
\clubpenalty=10000

\makeatletter 
\g@addto@macro\@verbatim\footnotesize 
\makeatother 

\begin{document}
\FrontMatter
\ThesisTitlePage

\begin{ThesisDeclaration}
\DeclarationText
\AdvisorName
\end{ThesisDeclaration}

\begin{ThesisAbstract}
TODO...
\end{ThesisAbstract}

\begin{ThesisKeyWords}
TODO...
\end{ThesisKeyWords}

\begin{ThesisThanks}
TODO...
\end{ThesisThanks}

\MainMatter
\tableofcontents
\chapter{Úvod}
Java je dnes jedným z najpoužívanejších programovacích jazykov.
Od syntakticky podobných programovacích jazykov ako napríklad C++, alebo C\# sa
líši prekladom zdrojových tried do medzikódu často označovaného ako
bajtkód (\textit{bytecode, p-code, portable code}).

Preklad a spustenie programu napísaných v programovacom jazyku Java prebieha v
nasledujúcich fázach:

\begin{enumerate}
\item Preklad do medzikódu: Java~compiler~\footnote{Najčastejšie využívaným
Javacompilerom je \textit{javac}, ktorý je súčasťou JDK (Java Development Kit).}
preloží zdrojový kód do bajtkódu. V praxy to znamená, že každej triede, alebo
rozhraniu je priradený súbor \textit{class}, ktorý obsahuje inštrukcie
popisujúce fungovanie danej triedy. \item Načítanie a Interpretácia: Virtuálny
stroj Javy (ďalej len JVM~\footnote{Java Virtual Machine, špecifikácia je
dostupná na \url{http://docs.oracle.com/javase/specs/jvms/se7/html}}) načíta
inštrukcie \textit{class} súboru potrebnej triedy, ktoré ďalej spracúva jedným
z nasledujúcich spôsobov:

\begin{itemize}
\item JIT prekladač (\textit{Just In Time compiler}): Štandardne je z
bajtkódu najskôr vygenerovaný strojový (\textit{machine code}) konkrétneho
zariadenia, ktorý je následne interpretovaný priamo vykonávaný
procesorom.
\item Java interpreter: Ďalším spôsobom spracovania bajtkódu je
využitie Java interpretru, ktorý bajtkódkód spracováva a sám interpretuje.
\end{itemize}
\end{enumerate}

Výhodou prekladu do bajtkódu je jeho a prenositeľnosť. Samotný bajtkód je
platformovo nezávislí. Program teda nieje nutné prispôsobovať jednotlivým
operačným systémom, ktoré sa líšia len v implementácií JVM.

\textit{Class} súbory obsahujúce bajtkód je možné za behu programu modifikovať.
Jednotlivé triedy a rozhrania aplikácie uložené v týchto súboroch podľa potreby
načítava JVM. Vkladanie nových metód a tried na úrovni bajtkódu, pred
načítaním \textit{class} súboru do JVM sa nazýva injekcia bajtkódu
(\textit{bytecode injection}, ďalej len BI). Pridávanie novej funkcionality
pomocou BI bez nutnosti zastavenia behu programu je často využívané pri
testovaní a trasovaní (\textit{tracing}) programov.

\begin{figure}[h]
  \centering
   \includegraphics[width=\textwidth]{JVM.png}
  \caption{Grafické znázornenie prekladu a spustenia programu, zdroj: vlastné
  spracovanie}
  \label{fig:jvm}
\end{figure}

\section{Cieľ práce}
TODO...

\section{Členenie práce}
TODO...

\chapter{Bajtkód}
Po preklade zdrojových kódov prekladačom \textit{javac} je každej
triede, prípadne rozhraniu programu priradený jeden \textit{class} súbor
popisujúci jej funkcionalitu.

Pri načítavaní \textit{class} súbru JVM dostane takzvaný prúd inštrukcií
bajtkódu (\textit{bytecode stream}) pre každú metódu triedy. V prípade volania
konkrétnej metódy za behu programu sú inštrukcie danej metódy vykonávané.
Každá z inštrukcí bajtkódu je reprezentovaná číselnou hodnotou nazývanou
\textit{opcode}. Zároveň má každá inštrukcia aj textovú podobu (\textit
{mnemonic}), ktorá je jej menom. V \textit{class} súboorch sú
inštrukcie uložené v numerickej podobe.

Táto kapitola popisuje formát \textit{class} súboru a následne stručne
charakterizuje inštrukčnú sadu bajtkódu.~\footnote{Nasledujúci text vychádza 
zo 4. až 6. kapitoly špecifikácie JVM~\cite{Lindholm:2013:JVM:2462629}.}

\section{Štruktúra \textit{class} súboru}
\label{sec:classFile}
\textit{Class} súbor pozostáva z jednej \textit{ClassFile} štruktúry. \textit
{ClassFile} štruktúra jednoznačne identifikuje konkrétnu triedu, prípadne
rozhranie, definuje jej premenné a metódy.

Nasledujúci popis definuje sadu datových typov. Typy \textit {u1},
\textit {u2}, a \textit {u4} reprezentujú neznamienkové jedno, dvoj, alebo
štvorbajtové číslo. \textit {ClassFile} je zobrazená ako pseudoštruktúra v
notácií jazyka C. Obsah štruktúry je popísaný ako po sebe nasledujúce položky.

\begin{example}{Formát \textit{ClassFile} štruktúry}
\begin{verbatim}
ClassFile {
  u4 magic;
  u2 minor_version;
  u2 major_version;
  u2 constant_pool_count;
  cp_info constant_pool[constant_pool_count-1];
  u2 access_flags;
  u2 this_class;
  u2 super_class;
  u2 interfaces_count;
  u2 interfaces[interfaces_count];
  u2 fields_count;
  field_info fields[fields_count];
  u2 methods_count;
  method_info methods[methods_count];
  u2 attributes_count;
  attribute_info attributes[attributes_count];
}
\end{verbatim}
\end{example}

Konštanta \textit{magic} identifikuje formát súboru \textit{class},
jej hodnota je 0xCAFEBABE.

Položky \textit{minor\_version} a \textit{major\_version}
určujú verziu \textit{class} súboru. Napríklad \textit{minor\_version} s
hodnotou \textit{m} a \textit{major\_version} s hodnotou \textit{M} indikujú
verziu s hodnotou \textit{M.m}.

Hodnota položky \textit{constant\_pool\_count} je rovná počtu záznamov
v \textit{constant\_pool[]} plus jeden.

Úložisko záznamov \textit{constant\_pool[]} (v podobe poľa štruktúr)
zahŕňa rôzne konštanty: mená tried a rozhraní, mená premenných a iné. Každý
záznam \textit{constant\_pool[]} sa skladá zo značky (\textit{tag}) a indexu
(\textit{name index}). Značka určuje typ záznamu. Tabuľka značiek je uvedená
v prílohe \ref{tab:tab1}. Pomocou unikátneho indexu, je možné odkazovať sa na
záznamy v ďalších častiach bajtkódu. Existuje niekoľko typov
štruktúr~\footnote {Všetky štruktúry \textit{constant\_pool[]} sú popísane v
špecifikácií JVM~\cite{Lindholm:2013:JVM:2462629}.} reprezentujúcich rôzne
druhy záznamov. Napríklad štruktúra \textit{CONSTANT\_String\_info}
reprezentuje objekty typu \textit{String} zatiaľ čo štruktúry
\textit{CONSTANT\_Methodref\_info} a \textit
{CONSTANT\_InterfaceMethodref\_info} reprezentujú metódy danej triedy, alebo
rozhrania.

Hodnota \textit{access\_flags} popisuje oprávnenia prístupu k
informáciam a vlastnosti tejto triedy, respektíve rozhrania pomocou
indikátorov. Napríklad nastavenie indikátora \textit{ACC\_INTERFACE} znamená,
že \textit{class} súbor popisuje rozhranie. Tabuľka indikátorov je uvedená v
prílohe \ref{tab:tab2}.

Položka \textit{this\_class} obsahuje index \textit{constant\_pool[]}
odkazujúci na štruktúru typu \textit{CONSTANT\_Class\_info}~\footnote
{\textit{CONSTANT\_Class\_info} je štrukura \textit{constant\_pool}, ktorá
reprezentuje triedu, alebo rozhranie.}. Reprezentuje triedu, respektíve
rozhranie, definované týmo class súborom.

Hodnotou \textit{super\_class} je taktiež index \textit{constant\_pool[]}
odkazujúci na štruktúru typu \textit{CONSTANT\_Class\_info}. Reprezentuje
priamu nadtriedu triedy definovanej týmto \textit{class} súborom. V prípade,
že tento \textit{class} súbor popisuje rozhranie, index odkazuje na triedu
\textit{Object}. Trieda \textit{Obejct} má ako jediná hodnotu
\textit{super\_class} nulovú.

Počet rozhraní, ktoré trieda implementuje vyjadruje položka
\textit{interface\_count}, v prípade rozhrania je táto položka rovná počtu
priamych nadrozhraní.

Pole \textit{interfaces[]} obsahuje indexy \textit{constant\_pool[]}
odkazujúce na štruktúru typu \textit{CONSTANT\_Class\_info}. Zahŕňa indexy
všetkých rozhraní, ktoré sú implementované triedou, prípadne priamymi
nadrozhraniami \textit{class} súboru.

Položka \textit{fields\_count} je rovná počtu premenných triedy a premenných
inštancí (\textit{fields}) \textit{class} súboru.

Štruktúry typu \textit{field\_info} sú združené v poli \textit{fields[]}. Toto
pole zahŕňa každú premennú danej triedy, respektíve rozhrania. Nezahŕňa
zdedené atribúty. Podrobne sa štruktúrou \textit{field\_info} sa zaoberá
kapitola \ref{sec:fields}.

Hodnata položky \textit{methods\_count} vyjadruje počet štruktúr
\textit{method\_info} v poli \textit{methods[]}.

Položka \textit{methods[]} je pole štruktúr typu \textit{method\_info}. Každá
štruktúra \textit{method\_info} popisuje metódu tejto triedy, respektíve
rozhrania. Zahŕňa konštruktory, metódy triedy a
metód inštancí. Neobsahuje však žiadne zdedené metódy. Štruktúru
\textit{method\_info} popisuje kapitola \ref{sec:methods}.

Hodnota \textit{attributes\_count} je rovná počtu atribútov poľa
\textit{attributes[]} \textit{class} súboru.

Pole \textit{attributes[]} obsahuje štruktúry typu \textit{attribute\_info}.
Atribútmi štruktúry \textit{ClassFile} sú napríklad: \textit{SourceFile},
\textit{Deprecated}, \textit{InnerClasses} a iné. Atribút \textit{SourceFile} 
slúži na reprezentáciu mena \textit{class} súboru. Pole \textit{attributes[]}
\textit{class} súboru môže obsahovať maximálne jeden takýto atribút. Atribút
\textit{Depricated} môže byť použitý v prípade, že bola daná trieda nahradená
(\textit{depricated}). Pri volaní takejto triedy môže prekladač upozorníť 
užívateľa, že sa odkazuje na nahradenú triedu~\footnote{Rovnakým spôsobom je
možné atribút \textit{Depricated} aplikovať aj na premenné a metódy.}. Vo
všeobecnosti sa štruktúre \textit{attribute\_info} sa venuje
kapitola \ref{sec:attributes}.

\subsection{Reprezentácia dátových typov}
\label{sec:descriptors}
Dátové typy sú v \textit{class} súboroch reprezentované vo formáte reťazcov
s kódovaním \textit{UTF-8}. Delíme ich na:
\begin{itemize}
\item dátové typy premenných
\begin{itemize}
\item primitívne dátové typy
\item referenčné dátové typy
\item polia
\end{itemize}
\item dátové typy metód
\end{itemize}

Primitívnym dátovým typom (\textit{byte}, \textit{integer}, …) je priradený
popis v podobe znaku (\textit{B}, \textit{I}, …). Napríklad premenná typu
\textit{int} je reprezentovaná znakom: \textit{I}.

Referenčné dátové typy reprezentuje popis v tvare: \textit{L<classname>;}, kde 
\textit{classname} je meno triedy, alebo rozhrania daného referenčného
dátového typu. Premenná typu \textit{Object} je interpretovaná ako
\textit{java/lang/Object;}. 

Identifikačný reťazec jednorozmerného poľa typu \textit{T} sa značí
\textit{[T}, pričom počet znakov \textit{[} je rovný dimenzii poľa. Napríklad
premenná typu: \textit{double d[][][]} generuje reťazec: \textit{[[[D}.

Reťazec dátového typu metódy sa skladá z reťazcov pre dátový typ parametrov,
ohraničených v zátvorkách \textit{(P*)} a reťazca pre dátový typ návratovej
hodnoty \textit{R}. Tvar reťazca dátového typu metódy je potom \textit{(P*)R}.
V prípade návratovej hodnoty \textit{null} je reťazcom návratovej hodnoty znak
\textit{V}. Napríklad metódu \textit{boolean long pow (int n, int k)}
reprezentuje reťazec: \textit{(II)J}, v prípade metódy
\textit{Object method(byte b)} by šlo o reťazec:
\textit{(B)Ljava/lang/Object;}. Komplexný prehľad reprezentácie datových typov
je uvedený v prílohe \ref{tab:tab3}.

\subsection{Premenné tried a inštancií}
\label{sec:fields}
Premenné tried inštancií (\textit{fields}) \textit{class} súboru sú v poli
\textit{fields[]} reprezentované pomocou štruktúry \textit{field\_info}.
Formát štruktúry \textit{field\_info} je nasledovný:

\begin{example}{Štruktúra \textit{field\_info}}
\begin{verbatim}
field_info {
  u2 access_flags;
  u2 name_index;
  u2 descriptor_index;
  u2 attributes_count;
  attribute_info attributes[attributes_count];
}
\end{verbatim}
\end{example}

Položka \textit{access\_flags} je indikátorom oprávnenia prístupu k danej
premennej. Mená indikátorov spolu s ich interpretáciou a hodnotou sú uvedené v
prílohe \ref{tab:tab4}.
     
Dvojbajtová hodnota \textit{name\_index} je index \textit{constant\_pool[]}
reprezentujúci meno premennej

Podobne ako \textit{name\_index} aj \textit{descriptor\_index} je dvojbajtová
položka odkazujúca sa na štruktúru v \textit{constant\_pool}. Na rozdiel od
mena premennej však popisuje datový typ premennej. Reprezentáciou datových
typov sa zaoberá kapitola \ref{sec:descriptors}.

Položka \textit{attributes\_count} vyjadruje počet atribútov v poli
\textit{attributes[]}.

Pole \textit{attributes[]} môže obsahovať ľubovoľné množstvo atribútov
popisujúcich premennú. Štruktúra reprezentujúca atribút je daná všeobecným
predpisom \textit{attributeq\_info}. Atribúty premenných musia byť
reprezentované jednou zo štruktúr \textit{ConstantValue}, \textit{Synthetic},
\textit{Signature}, \textit{Deprecated}, \textit{RuntimeVisibleAnnotations},
alebo \textit{RuntimeInvisibleAnnotations}. Atribút \textit{ConstantValue}
popisuje konštantné statické premenné, \textit{Synthetic} je používaný u
položiek, ktoré sa nevyskytujú v zdrojovom kóde. Štruktúrou
\textit{attribute\_info} sa zaoberá kapitola \ref{sec:attributes}.

\subsection{Metódy}
\label{sec:methods}
Každá metóda triedy, prípadne rozhrania je v poli \textit{methods[]} uložená
pomocou štruktúry \textit{method\_info}. Štruktúra \textit{method\_info} má
nasledujúci formát:

\begin{example}{Štruktúra \textit{method\_info}}
\begin{verbatim}
method_info {
  u2 access_flags;
  u2 name_index;
  u2 descriptor_index;
  u2 attributes_count;
  attribute_info attributes[attributes_count];
}
\end{verbatim}
\end{example}

Indikátor \textit{access\_flags} zahŕňa nastavenia prístupových práv a
vlastností metódy. Tabuľaka indikátorov \textit{access\_flags} štruktúry
\textit{method\_info} sa nachádza v prílohe \ref{tab:tab5}.

Položky \textit{name\_index} a \textit{descriptor\_index} sú podobne ako u
štruktúry \textit{field\_info} indexmi do \textit{constant\_pool}. Tieto indexy
v \textit{constant\_pool} odkazujú na štruktúry popisujúce meno a datový typ
metódy. Reprezentácia dátových typov je popísaná v kapitole
\ref{sec:descriptors}.

Hodnotou položky \textit{attributes\_count} je počet atirbútov
poľa \textit{attributes[}.

Pole \textit{attributes[]} zahŕňa dodatočné atribúty (položky) danej metódy.
Každá položka poľa je reprezentovaná všeobecným predpisom
\textit{attributes\_info}. Počet štruktúr v poli nieje obmedzený, každá
položka však musí byť jednou zo štruktúr: \textit{Code}, \textit{Exceptions},
\textit{Synthetic},\textit{Signature}, \textit{Deprecated},
\textit{RuntimeVisibleAnnotations}, \textit{RuntimeInvisibleAnnotations},
\textit{RuntimeVisibleParameterAnnotations},
\textit{RuntimeInvisibleParameterAnnotations},
alebo \textit{AnnotationDefault}.
Atribút \textit{Code} je jedným z najdôležitejších. Obshauje inštrukcie
bajtkódu popisujúce fungovanie metódy. Okrem metód deklarovaných ako
abstraktná, alebo natívna musí každá metóda obsahovať práve jeden atribút
\textit{Code}. Atribút \textit{Exceptions} zahŕňa indexy výnimiek, ktoré
metóda vyhadzuje. Popisom formátu štruktúry \textit{attributes\_info} sa
zaoberá kapitola \ref{sec:attributes}.

\subsection{Atribúty}
\label{sec:attributes}
Pojem atribút v tomto texte vyjadruje atribúty používané v poli
\textit{attributes[]} štruktúr \textit{field\_info}, \textit{method\_info} a
\textit{Code\_attributes}. Všeobecný predpis všetkých atribútov je vyjadrený
štruktúrou \textit{attribute\_info}. Existuje niekoľko základných
preddefinovaných atribútov: \textit{SourceFile}, \textit{ConstantValue},
\textit{Code}, \textit{Exceptions}, \textit{InnerClasses}, \textit{Synthetic},
\textit{LineNumberTable}, \textit{LocalVariableTable}, \textit{Deprecated} a
iné. Líšia sa funkcionalitou a využitím jednotlivými časťami \textit{class}
súboru. Všetky atribúty vychádzajú z už spomínaného všeobecného predpisu
\textit{attribute\_info}, ktorý má nasledujúci formát:

\begin{example}{Štruktúra \textit{attribute\_info}}
\begin{verbatim}
attribute_info {
  u2 attribute_name_index;
  u4 attribute_length;
  u1 info[attribute_lenght];
}
\end{verbatim}
\end{example}

Položka \textit{attributes\_name\_index} je indexom do \textit{constant\_pool}
odkazujúcim na meno atribútu. Tento proces sa nazýca kontrola formátu
(\textit{format checking}). Prvé štyri bajty musia obsahovať tzv. magickú
konštantu \textit{magic}. Všetky rozoznané atribúty musia mať správnu dĺžku 

\section{Inštrukcie JVM}
Po načítaní \textit{class} súboru JVM sa JVM nasjkôr uistí, že je tento súbor
v správnom formáte popísanom v kapitole \ref{sec:classFile}. Štvorbajtová
položka \textit{attribute\_length} je rovná hodnote vyjadrujúcej
dĺžku následných informácií uložených v \textit{info[attribute\_length]}.
Informácie sa líšia na základe odlišnej funkcionality a využitá jednotlivých
atribútov. \textit{Class} súbor nesmie byť skrátený ani obsahovať nadbytočné
bajty, takisto úložisko \textit{constant\_pool} nesmie obsahovať žiadne
nerozoznateľné informácie.

Inštrukcie bajtkódu načítanej metódy sú uložené v poli \textit{code[]}
atribútu \textit{Code}, štruktúry \textit{method\_info} daného \textit{class}
súboru. Štruktúra \textit{Code\_attribute} reprezentujúca atribút \textit{Code}
musí spĺňať obmedzenia definované JVM. Tieto obmedzenie rozdeľujeme na dve
základné kategórie: 

\begin{itemize}
\item Statické obmedzenia: Stanovujú rozloženie inštrukcií v poli \textit{code}
a priradenie operandov jednotlivým inštrukciám. Niektorými z nich sú napríklad:
\begin{itemize}
\item prvá inštrukcia musí začínať na indexe 0,
\item pole \textit{code} nesmie byť prázdne.
\end{itemize}
\item Štrukturované obmedzenia: Špecifikujú vztahy medzi inštrukciami JVM. Ide
o podmienky ako napríklad:
\begin{itemize}
\item žiadna lokálna premenná nemôže byť volaná predtým ako jej bola priradená
hodnota,
\item pred volaním (nestatickej) metódy respektíve premennej musí byť
inicializovaná inštancia triedy ktorá ju obsahuje.
\end{itemize}
\end{itemize}

Prekladače jazyka Java generujú \textit{class} súbory, ktoré spĺňajú požiadavnky
popísané v predchádzajúcom odseku. JVM však nemá žiadnu záruku, že
\textit{class} súbor, ktorý požaduje bol generovaný prekladačom. Metódou
verifikácie~\footnote{Ďalšie príklady obmedzení a podorbný popis verifikácie
\textit{class} súborov je dostupný v špecifikácií
JVM~\cite{Lindholm:2013:JVM:2462629}} \textit{class} súboru môže JVM určiť či
daný súbor pochádza z dôveryhodného zdroja.

\subsection{Dátové typy}
Dátové typy JVM delíme do troch základných
kategórií:

\begin{itemize}
\item Primitívne dátové typy: byte, short, int, long, boolean, float, double.
\item Referenčné dátové typy: pole, inštancia triedy, rozhranie.
\item Typ \textit{returnAddress}: používaný výhradne ištrukciami
\textit{jsr}, \textit{ret} a \textit{jsr\_w}.
\end{itemize}

Väčšina uvedených typov má veľkosť 32 bitov, typy \textit{long} a
\textit{double} sú však 64 bitové, preto zaberajú dva sloty v zásobníku.

\subsection{Architektúra a inštrukčná sada}
Architektúra JVM je založená na datovej~štruktúre~zásobník~\footnote{Dátová
štruktúra zásobník (\textit{stack}) funguje na princípe FIFO (\textit{first in
first out}), kde posledný vložený prvok je prvým vybraným.}, ktorej základnými
operáciami sú \textit{push} - vloženie prvku do zásobníka a \textit{pop} - výber
prvku z vrcholu zásobníka. JVM nemá registre na ukladanie hodnôt, preto musia
byť pred použitím všetky uložené na zásobník.

Na nasledujúcom jednoduchom príklade sú popísané základné inštrukcie bajtkódu
pre prácu s premennými a konštantami.

\begin{example}{Metóda \textit{greatherThen} pred a po kompilácií}
\begin{verbatim}
public int greaterThen(int intOne, int intTwo) {
  if (intOne > intTwo) {
    return 0;
  } else {
    return 1;
  }
}

0: iload_1
1: iload_2
2: if_icmple  7
5: iconst_0
6: ireturn
7: iconst_1
8: ireturn
\end{verbatim}
\end{example}

Inštrukcie \textit{iload\_1} a \textit{iload\_2} pridajú do zásobníka operandov 
(ďalej len zásobník) hodnoty lokálnych premenných na indexoch 1 a 2. V tomto 
prípade ide o parametre \textit{intOne} a \textit{intTwo}. 

\begin{figure}[H]
  \begin{minipage}{0.55\textwidth}
     \includegraphics[width=0.825\textwidth]{iload_1.png}
  \end{minipage}
  \begin{minipage}{0.55\textwidth}
     \includegraphics[width=0.825\textwidth]{iload_2.png}
  \end{minipage}
  \caption{Znázornenie funkcionality inštrukcií \textit{iload\_1} a
  \textit{iload\_2}.}
  \label{fig:gTiload}
\end{figure}

Vo všeobecnosti môžeme túto inštrukciu chápať ako \textit{xload} s predponou \t
extit{x} označujúcou ľubovoľný primitívny datový typ(napríklad: \textit{lload} 
pre long, \textit{fload} pre float). Existujú dva tvary, volania tejto 
inštrukcie: 

\begin{itemize}
\item \textit{load\_<n>}, kde \textit{n} označuje index (celé číslo) lokálnej 
premennej, zároveň musí platiť: $0 \leq n \leq 4$,
\item \textit{load vindex}, kde pozíciou lokalnej premennej je hodnota
\textit{vindex}.
\end{itemize}

Ďalšou inštrukciou je \textit{if\_icmple} s parametrom 7, ktorá porovná dva 
objekty na vrchole zásobníka a prejde na siedmu inštrukciu v prípade, že je 
hodnota položky na vrchole zásobníka väčšia ako hodnota druhej položky. V 
príklade sú na zásobníku len položky vložené predchádzajúcimi inštrukciami.
Podmienka teda platí v prípade, že hodnota parametra \textit{intOne} je menšia 
hodnota \textit{intTwo}. Vo všeobecnosti je možné podmienené výrazy vyjadriť 
pomocou inštrukcií: \textit{if\_acmp<cond>, if\_icmp<cond>, if<cond>, 
ifnonnull, ifnull}.

Inštrukcie \textit{iconst\_0} a \textit{iconst\_1} vložia na zásobník hodnotu 0
respektíve 1 v závislosti od vyhodnotenia podmienky \textit{if\_icmple}. Táto 
hodnota je následne vrátená inštrukciou \textit{ireturn}. Inštrukcie
\textit{iconst\_<n>, a ireturn} sú taktiež dostupné vo variantách s predponou 
ľubovoľného primitívneho dátového typu.

Dôkladný popis všetkých inštrukcií vrátene ich parametrov možno nájsť v 
špecifikácií JVM~\cite{Lindholm:2013:JVM:2462629}.

\chapter{Classloadery} 
\textit{Class loader} je objekt zodpovedný za načítavanie tried. Trieda
\textit{ClassLoader} je abstraktná. Pomocou mena \textit{class} súboru by mal
\textit{class loader} nájsť a generovať obsah definujúci danú triedu. Každá
trieda obsahuje referenciu na \textit{ClassLoader}, ktorý ju
definoval.~\cite{Oracle:ClassLoader}

Zvyčajne je trieda do JVM načítaná len v prípade, že je potrebná. Načítané sú 
zároveň všetky triedy na ktoré sa odkazuje. Pomocou \textit{class loaderov} je
možné za behu programu dynamicky načítať ďalšie triedy, prípadne načítať nové 
inštancie pôvodných tried.

Pri štandardnom načítaní triedy niektorá z implementácií \textit{ClassLoader} 
vykoná nasledujúce tri kroky:
\begin{enumerate}
\item Skontroluje či trieda už nebola načítaná
\item Ak nebola, požiada nadtriedu o načítanie danej triedy 
\item V prípade, že nuspeje pokúsi sa načítať triedu pomocou
vlastného \textit{class loaderu}
\end{enumerate}

\subsection{Dynamické načítavanie tried}
K načítaniu novej triedy za behu programu je potrebný \textit{class loader}.
Získať ho je možné pomocou príkazu \textit{MyClass.class.getClassLoader();}.
Novú triedu reprezentovanú súborom \textit{class} následne vráti metóda
\textit{class loaderu}: \textit{loadClass(class)}.

\subsection{Znovunačítanie triedy}
Dynamické znovunačítanie triedy je komplkovanejšie. Vstavané implementácie 
triedy \textit{ClassLoader} vždy kontrolujú, či trieda už nebola do JVM
načítaná. Preto nieje možné žiadnu triedu načítať dvakrát pomocou vstavaných
\textit{class loaderov}. Je nutné navrhnúť vlastnú implementáciu.

-

! PRÍKLAD VLASTNÉHO CLASSLOADERA - TODO ... !

-

Ďalšou komplikáciou je trieda \textit{ClassLoader.resolve()},
ktorá zabezbečuje linkovanie. Táto trieda je \textit{final}, z čoho vyplíva, že 
ju nieje možné prepísať, nepovolí však žiadnemu \textit{class loaderu} linkovať
dva-krát tú istú triedu. Preto je nutné pri kaďom ďalšom znovunačítaní triedy
vytvoriť novú inštanciu \textit{class loaderu}. %Štruktúra implementácie vlastného \textit{class loaderu} je uvedená v prílohe \ref{ex:cLoader}.

-

! PRÍKLAD POUŽITIA CLASSLOADERA - TODO ... !

\chapter{Byteman}
\textit{Byteman} je nástroj manipulujúci s bajtkódom určný na zásah do bajtkódu Java aplikácií počas načítavania JVM, alebo za behu programu. 
Používa sa najmä na zjednodušenie trasovania a testovania aplikácií. Umožňuje 
používateľovi pridávať novú funkcionalitu do ktorejkoľvek časti programu.
Funguje bez nutnosti prepisovania a opätovnej kompilácie pôvodnej aplikácie.

\textit{Byteman} modifikuje bajtkód aplikácie za behu programu. Preto môže  
zmeniť Java kód, popisujúci časť treid JVM ako napríklad \textit{String},
\textit{Thread} a podobne. Vďaka tejto funkcionalite je taktiež možné
napríklad trasovanie správania sa JVM.

\textit{Byteman} používa jednoduchý jazyk
ECA~pravdiel~\footnote{ECA (\textit{event-condition-action}) pravidlá 
pozostávajú z udalosti, podmienky a akcie. Význam pravidla znamená: Ak nastane 
udalosť, skontrolu podmienku a v prípade, že platí, vykonaj
akciu~\cite{Sellis:ECARules}.} založený na Jave. Tieto ECA pravidlá
používa na špecifikáciu kde, kedy a ako má byť pôvodný Java kód transformovaný
aby modifikoval operáciu~\cite{RedHat:Byteman}.

Primárne bol \textit{Byteman} určený na podoru testovania multivláknových a
multi-JVM aplikácií za použitia techniky nazývanej
\textit{fault~injection~\footnote{TODO...}}. Zahŕňa preto funkcionalitu, ktorá 
bola navrhnutá na riešenie problémov súvisiacich s týmto typom testovania.
\textit{Byteman} poskytuje podporu pre automatizáciu v štiroch hlavných 
oblastiach:

\begin{itemize}
\item trasovanie špecifických väzieb kódu a zobrazovanie stavu 
aplikácie, prípadne JVM,
\item narúšanie normálneho priebehu zmenou stavov, volanie nenaplánovaných
metód, vynucovanie návratových volaní, prípadne vyhadzovanie neočakávaných 
výnimiek,
\item organizácia časovania aktivít vykonaných nezávislými vláknami aplikácie,
\item monitorovanie a zhromažďovanie štatistík, sumarizujúcich aplikáciu a
operácie JVM.
\end{itemize}

V súčasnosti je \textit{Byteman} využívaný oveľa širšie ako nástroj na
testovanie~\cite{RedHat:Byteman}. 

Najjednoduchším použitím \textit{Bytemana} je vkladanie kódu, ktorý trasuje 
správanie sa aplikácie. Táto metóda môže byť využitá na monitorovanie,
alebo ladenie, ako aj na úpravu kódu pri testovaní a overenie, správneho 
fungovania aplikácie. Pri vkladaní kódu na veľmi špecifické miesta je možné
vyhnúť sa režijným nákaldom, ktoré často rastú pri ladení, alebo trasovaní 
proguktu~\cite{Byteman:Homepage}.

\section{Byteman Agent}

Aby mohol \textit{Byteman} manipulovať s programom, musí na ňom bežať
\textit{Byteman Agent}, ktorý konfiguruje JVM pre prácu s pravidlami jeho jazyka.

Pri inštalácií agenta s prekladom programu je riešením použitie argumentu príkazu
\textit{java -javaagent}, ktorý zadáva cestu k \textit{JAR} súboru popisujúcemu 
pravidlá jazyka. Agentovi je možné pomocou argumentov nakonfigurovať dve
základné možnosti funkcionality:

\begin{itemize}
\item Základnou možnosťou je použiť argument \textit{script:[PATH]}, ktorý
načíta do programu skript definovaný pravidlami v súbore s cestou
\textit{PATH}
\item V prípade potreby načítavania pravidiel do programu aj po spustení je nutné
nastaviť argument \textit{listener} na hodnotu \textit{true}. Do takto
spusteného programu je možné následne pomocou skriptu \textit{bmsubmit.sh}
pridávať a odoberať ľubovoľné pravidlá.
\end{itemize}

\textit{Byteman} je nastavený aby neinjektoval kód do tried JVM. Pri zmene tried
ako napríklad \textit{String a Thread} je preto nutné zmeniť túto vlastnosť
nastavením \textit{system property org.jboss.byteman.transform.all}. Zároveň je 
nutné zaistiť, aby bol \textit{Byteman Agent} načítaný (rovnako ako tieto triedy)
defaultným (\textit{bootstrap}) \textit{classloaderom}.

Agenta je možné inštalovať taktiež do už bežaicich aplikácií \footnote{typicky
ide o dlho bežiace aplikácie ako napríklad aplikačný server JBoss} bez nutnosti 
ich opätovného spustenia. Slúži na to skipt \textit{bminstall.sh}.
\textit{Byteman} je následne možné využiť ako nástroj na kontrolu správania sa 
programu~\cite{RedHat:Byteman}. 

\section{Jazyk}
TODO...

\chapter{Nástroj Javasist}
TODO...

\clearpage
\addcontentsline{toc}{chapter}{Literatúra} 
\bibliographystyle{alpha} 
\bibliography{bp} 

\appendix

\chapter{Tabuľky}
Zdrojom nasledujúcich tabuliek je špecifikácia
JVM~\cite{Lindholm:2013:JVM:2462629}.

\begin{table}
  \begin{tabular}{| l | c |}
    \hline
    \textbf{Constant Type} & \textbf{Value} \\
    \hhline{|=|=|}
    CONSTANT\_Class & 7 \\ \hline
    CONSTANT\_Fieldref & 9 \\ \hline
    CONSTANT\_Methodref & 10 \\ \hline
    CONSTANT\_InterfaceMethodref & 11 \\ \hline
    CONSTANT\_String & 8 \\ \hline
    CONSTANT\_Integer & 3 \\ \hline
    CONSTANT\_Float & 4 \\ \hline
    CONSTANT\_Long & 5 \\ \hline
    CONSTANT\_Double & 6 \\ \hline
    CONSTANT\_NameAndType & 12 \\ \hline
    CONSTANT\_Utf8 & 1 \\ \hline
    CONSTANT\_MethodHandle & 15 \\ \hline
    CONSTANT\_MethodType & 16 \\ \hline
    CONSTANT\_InvokeDynamic & 18 \\
    \hline
  \end{tabular}
  \caption{Tabuľka značiek určujúcich typ záznamu v \textit{constant\_pool}.
  Stĺpec \textit{Constant Type} označuje názov typu, stĺpec \textit{value}
  priraďuje každému typu číselnú hodnotu.}
  \label{tab:tab1}
\end{table}

\begin{table}
  \begin{tabular}{| l | c | p{6cm} |}
    \hline
    \textbf{Meno Indikátora} & \textbf{Hodnota} & \textbf{Interpretácia} \\
    \hhline{|=|=|=|}
    ACC\_PUBLIC & 0x0001 & Deklarovaná ako verejná; prístupná aj mimo balíka.
    \\ \hline
    ACC\_FINAL & 0x0010 & Deklarovaná ako final; žiadne podtriedy po 
    inicializácií. \\ \hline
    ACC\_SUPER & 0x0020 & Volá metódu nadtriedy, hlavne inštrukcia 
    invokespecial. \\ \hline
    ACC\_INTERFACE & 0x0200 & Je rozhranie, nie trieda.\\ \hline
    ACC\_ABSTRACT & 0x0400 & Deklarovaná ako abstraktná, nemôže byť 
    inštanciovaná. \\ \hline
    ACC\_SYNTHETIC & 0x1000 & Deklarovaná ako synthetic, nieje prítomná v 
    zdrojovom kóde. \\ \hline
    ACC\_ANNOTATION & 0x2000 & Deklarovaná ako typ annotation. \\ \hline
    ACC\_ENUM & 0x4000 & Deklarovaná ako typ enum. \\    
    \hline
  \end{tabular}
  \caption{Tabuľka indikátorov prístupových práv \textit{ClassFile} štruktúry.}
  \label{tab:tab2}
\end{table}

\begin{table}
  \begin{tabular}{| p{3cm} | c | p{6,5cm} |}
    \hline
    \textbf{Reprezentácia pomocou reťazca} & \textbf{Typ} & 
    \textbf{Interpretácia} \\
    \hhline{|=|=|=|}
     B & byte &  znamienkové celé číslo veľkosti jedného bajtu \\ \hline
     C & char & Znak s kódovaním UTF-16 \\ \hline
     D & double & číselná hodnota s dvojitou presnosťou a plávajúcou
     desatinnou čiarkou \\ \hline
     F & float & číselná hodnota s plávajúcou desatinnou čiarkou \\ \hline
     I & int & celé číslo \\ \hline
     J & long & celé číslo väčšieho rozsahu \\ \hline
     L ClassName ; & referencia & inštancia triedy ClassName \\ \hline
     S & short & znamienkové celé číslo krátkeho rozsahu \\ \hline
     Z & boolean & pravda alebo nepravda \\ \hline
     [ & reference & jednorozmerné pole \\
    \hline
  \end{tabular}
  \caption{Tabuľka reprezentácie datových typov pre premenné.}
  \label{tab:tab3}
\end{table}

\begin{table}
  \begin{tabular}{| l | c | p{6,5cm} |}
    \hline
    \textbf{Meno Indikátora} & \textbf{Hodnota} & \textbf{Interpretácia} \\
    \hhline{|=|=|=|}
    ACC\_PUBLIC & 0x0001 & Deklarovaná ako verejná; prístupná aj mimo balíka.
    \\ \hline
    ACC\_PRIVATE & 0x0002 & Deklarovaná ako privátna; použiteľná len vrámci 
    triedy, v ktorej bola definovaná. \\ \hline
    ACC\_PROTECTED & 0x0004 & Deklarovaná ako protected; prístupná aj 
    podtriedam. \\ \hline
    ACC\_STATIC & 0x0008 & Deklarovaná ako statická. \\ \hline
    ACC\_FINAL & 0x0010 & Deklarovaná ako final; žiadne ďalšie priradenia po
    inicializácií. \\ \hline
    ACC\_VOLATILE & 0x0040 & Deklarovaná ako volatile; nemôže byť uložená do
    medzipamäte. \\ \hline
    ACC\_TRANSIENT & 0x0080 & Deklarovaná ako transient; nieje čítaná ani
    modifikovaná objektovým manažérom. \\ \hline
    ACC\_SYNTHETIC & 0x1000 & Deklarovaná ako synthetic, nieje prítomná v
    zdrojovom kóde. \\ \hline
    ACC\_ENUM & 0x4000 & Deklarovaná ako prvok objektu enum \\
    \hline
  \end{tabular}
  \caption{Tabuľka indikátorov prístupových práv a vlastností štruktúry
  \textit {field\_info}.}
  \label{tab:tab4}
\end{table}

\begin{table}
  \begin{tabular}{| l | c | p{5,5cm} |}
    \hline
    \textbf{Meno Indikátora} & \textbf{Hodnota} & \textbf{Interpretácia} \\
    \hhline{|=|=|=|}
    ACC\_PUBLIC & 0x0001 & Deklarovaná ako verejná; prístupná aj mimo balíka.
    \\ \hline
    ACC\_PRIVATE & 0x0002 & Deklarovaná ako privátna; použiteľná len vrámci 
    triedy, v ktorej bola definovaná. \\ \hline
    ACC\_PROTECTED & 0x0004 & Deklarovaná ako protected; prístupná aj 
    podtriedam. \\ \hline
    ACC\_STATIC & 0x0008 & Deklarovaná ako statická. \\ \hline
    ACC\_FINAL & 0x0010 & Deklarovaná ako final; nemôže byť prepísaná.
    \\ \hline
    ACC\_SYNCHRONIZED & 0x0020 & Deklarovaná ako synchronized; pri volaní je
    zabalená za použitia monitora. \\ \hline
    ACC\_BRIDGE & 0x0040 & Bridge metóda; je generovaná prekladačom. \\ \hline
    ACC\_VARARGS & 0x0080 & Deklarovaná s dynamickým počtom argumentov.
    \\ \hline
    ACC\_NATIVE & 0x0100 & Deklarovaná ako natívna; implementovaná v inom
    jazyku ako Java. \\ \hline
    ACC\_ABSTRACT & 0x0400 & Deklarovaná ako abstraktná, nieje implementovaná.
    \\ \hline
    ACC\_STRICKT & 0x0800 & Deklarovaná ako stricktfp, výpočty s plávajúcou
    čiarkou sú FP - strict. \\ \hline
    ACC\_SYNTHETIC & 0x1000 & Deklarovaná ako synthetic, nieje prítomná v
    zdrojovom kóde. \\
    \hline
  \end{tabular}
  \caption{Tabuľka indikátorov prístupových práv a vlastností štruktúry
  \textit {method\_info}.}
  \label{tab:tab5}
\end{table}

%\begin{example}{Príklad implementácie vlastného \textit{class loaderu}.}
%\begin{verbatim}
%class NetworkClassLoader extends ClassLoader {
%  String host;
%  int port;
%
%  public Class findClass(String name) {
%    byte[] b = loadClassData(name);
%    return defineClass(name, b, 0, b.length);
%  }
%
%  private byte[] loadClassData(String name) {
%    // load the class data from the connection
%    …
%  }
%}
%\end{verbatim}
%\caption{Príklad zachytáva štruktúru vlastnej implementácie triedy
%\textit{ClassLoader}. \textit{NetworkClassLoader} by mohol slúžiť napríklad na
%načítanie dát zo siete. Zdroj: \cite{Oracle:ClassLoader}.}
%\label{ex:cLoader}
%\end{example}


\end{document}